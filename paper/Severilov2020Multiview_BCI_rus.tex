\documentclass[12pt]{article}
\usepackage[utf8]{inputenc}
\usepackage[english,russian]{babel}
\usepackage{a4wide}
\usepackage{graphicx}
\usepackage{amssymb}
\usepackage{amsmath}
\usepackage{color}
\usepackage{url}
\usepackage{tikz}
\usetikzlibrary{matrix}

\usepackage[numbers,sort&compress]{natbib}

\DeclareMathOperator*{\argmax}{arg\,max}
\DeclareMathOperator*{\argmin}{arg\,min}
\newcommand*{\No}{No.}

\usepackage[pdftex,unicode, 
colorlinks=true,
linkcolor = blue
]{hyperref}	% нумерование страниц, ссылки!!!!ИМЕННО В ТАКОМ ПОРЯДКЕ СО СЛЕДУЮЩИМ ПАКЕТОМ
\newcommand{\hdir}{.}

\newcommand{\bx}{\mathbf{x}}
\newcommand{\by}{\mathbf{y}}
\newcommand{\bw}{\mathbf{w}}
\newcommand{\ba}{\mathbf{a}}
\newcommand{\bz}{\mathbf{z}}
\newcommand{\bb}{\mathbf{b}}
\newcommand{\bY}{\mathbf{Y}}
\newcommand{\bX}{\mathbf{X}}
\newcommand{\bu}{\mathbf{u}}
\newcommand{\bt}{\mathbf{t}}
\newcommand{\bp}{\mathbf{p}}
\newcommand{\bq}{\mathbf{q}}
\newcommand{\bc}{\mathbf{c}}
\newcommand{\bP}{\mathbf{P}}
\newcommand{\bT}{\mathbf{T}}
\newcommand{\bB}{\mathbf{B}}
\newcommand{\bQ}{\mathbf{Q}}
\newcommand{\bC}{\mathbf{C}}
\newcommand{\bE}{\mathbf{E}}
\newcommand{\bF}{\mathbf{F}}
\newcommand{\bU}{\mathbf{U}}
\newcommand{\bW}{\mathbf{W}}


\newcommand{\T}{^{\text{\tiny\sffamily\upshape\mdseries T}}}

\usepackage{graphicx}


\newtheorem{definition}{Определение}[section]

\usepackage{subcaption}
\usepackage{neuralnetwork}


\begin{document}
	\title{BCI\thanks{no}}
	\date{}
	\author{}
	\maketitle
	
	\begin{center}
		\bf
		П.\,А.~Северилов\footnote{Московский физико-технический институт, severilov.pa@phystech.edu}, 
		В.\,В.~Стрижов\footnote{Вычислительный центр имени А.\,А.\,Дородницына Федерального исследовательского центра <<Информатика и управление>> Российской академии наук, Московский физико-технический институт, strijov@phystech.edu}
	\end{center}
	{\begin{quote}
			\textbf{Аннотация:}
			В работе исследуется задача 
			
			\smallskip
			\textbf{Ключевые слова}: 
			\smallskip
			
			\textbf{DOI}: 00.00000/00000000000000
		\end{quote}
	}

	
	\section{Введение}
	В данной работе решается задача 
	
	\section{Постановка задачи}
	
	Пусть дана выборка $(\bX, \bY)$, где $\textbf{X} = [\textbf{x}_1, \dots, \textbf{x}_{n}]^{\T} \in \mathbb{R}^{n \times m}$~--- матрица независимых переменных, $\textbf{Y} = [\textbf{y}_1, \dots, \textbf{y}_n]^{\T} \in \mathbb{R}^{n \times k}$~--- матрица целевых переменных.
	
	
	\subsection{Метод частичных наименьших квадратов}
	

	\subsection{Канонический анализ корреляций}
	
	Канонический анализ корреляций
	
	\subsection{Нелинейный канонический анализ корреляций}
	
	Нелинейный канонический анализ корреляций~--- нелинейная модификация CCA. Метод Deep CCA преобразует исходные данные с помощью нейронной сети таким 

	
	\section{Вычислительный эксперимент}
	Целью вычислительного эксперимента является 
	В рамках вычислительного эксперимента написан программный комплекс для решения поставленных задач~\cite{source_code}.

	


	\section{Заключение}
	В работе рассмотрена задача 
	
	\bibliographystyle{unsrt}
	\begin{thebibliography}{99}
		
	
		\bibitem{source_code}
		\textit{Severilov}. Project source code is available at:~\url{https://github.com/severilov/BCI-thesis}, 2021.
	\end{thebibliography}
	
\end{document}

